\chapter{GPU Computing}
\label{sec:gpu}

Questo capitolo fornisce un'introduzione al GPU computing, per rendere il progetto di tesi più fruibile al lettore. Il capitolo è composto da ... sezioni:

\begin{itemize}
    \item breve storia di come le GPU sono diventate importanti in ambito scientifico
    \item architettura di una GPU e differenze con le CPU
    \item metodi di programmazione con delle GPU
\end{itemize}

\section[Breve storia delle GPU in ambito scientifico]{Storia delle GPU in ambito scientifico}

Le GPU sono state originariamente sviluppate per scopi legati all'elaborazione grafica, in particolare per migliorare la resa delle immagini e la grafica 3D nei videogiochi, nelle applicazioni CAD e nei software di modellazione e rendering. Le esigenze di una grafica sempre più dettagliata e complessa hanno portato alla creazione di unità di elaborazione specializzate in grado di gestire gli intensi calcoli in modo parallelo. Con il tempo è emerso che le GPU potevano essere sfruttate per scopi diversi dall'elaborazione grafica: i loro processori palalleli si sono rivelati utili per applicazioni scientifiche e tecniche che richiedavano elaborazione intensive, come la simulazione, l'analisi dei dati e il calcolo scientifico. Adesso in particolare le GPU si sono rivelate particolarmente adatte per l'addestramento e l'implementaione di reti neurali, diventando uno strumento fondamentale per l'esplosione dell'intelligenza atrificiale e del machine learning.

% GPGPU

Quando si parla di GPGPU si intende l'utilizzo delle GPU per compiti di calcolo generale, rendendole molto più che semplici dispositivi per l'elaborazione grafica. Questo è fondamentale per accelerare algoritmi che richiedono enorme quantità di risorse, unicamente a patto che la computazione sia parallelizzabile su più processori.
Calcoli che prima richiedevano l'uso di supercomputer per essere eseguiti in tempi accettabili adesso si possono eseguire in una normale GPU desktop. 
Nel 2011 secondo la TOP500 (https://www.top500.org/ classifica che descrive nel dettaglio i 500 sistemi informatici non distribuiti più potenti al mondo) il Tianhe-1A, montando delle GPU Nvidia Tesla M2050 (todo: mettere link), si posizionava secondo raggiungendo i 4.7 petaFLOPS di velocità. 
Nel 2012 invece in primo supercomputer al mondo divenne il Titan (todo: aggiustare link https://nvidianews.nvidia.com/news/nvidia-powers-titan-world-s-fastest-supercomputer-for-open-scientific-research-6622738) montando 18,688 GPU Nvida Tesla K20X con altrettante CPU AMD Opteron raggiungendo i 17.59 petaFLOPS.
Da quell'anno in poi divenne chiaro che le GPU erano diventate un componente essenziale nell'HPC quanto nel desktop computing e dato che il supercomputing il motore trainante di molte delle tecnologie che vediamo nei processori moderni, si è sviluppato un circolo virtuolso per cui la necessità di processori sempre più veloci per elaborare dataset sempre più grandi, porta l'industria a produrre computer sempre più potenti. Ad oggi si sta delineando una divisione netta nella produzione di GPU per uso desktop e per uso scientifico, i principali produttori, quali AMD, Nvidia e, più recentemente, Intel, rilasciano prodotti ottimizzati per l'uno o per l'altro mercato, con lo scopo di soddisfare i requisiti richiesti da ogni settore. A giugno 2023 il supercomputer più veloce al mondo è il Frontier con 9,472 CPU AMD a 64 core e 37,888 GPU Radeon Instinct MI250X che riesce a raggiungere la velocità di 1.67 exaFLOPS, diventando il primo exacale supercomputer. 

% architettura gpu

\section[Architettura hardware]{Architettura hardware}

Le GPU si distinguono dalle CPU tradizionali per la loro capacità di gestire compiti altamente paralleli e intensi dal punto di vista computazionale. Le principali differenze a livello architetturale sono: 

\begin{itemize}
    \item migliaia di core indipendenti
    \item memoria globale per archiviare dati e istruzioni
    \item memoria locale di ogni core per archiviazione temporanea dei dati
    \item unità di controllo per coordinare le operazione di calcolo e il flusso dati tra memoria e core, distribuendo i carichi di lavoro
    \item architettura SIMD che consente di eseguire la stessa istruzione su più dati contemporaneamente
\end{itemize}

Per scrivere software per GPU, vengono utilizzate API, come CUDA o OpenCL. Queste API consentono ai programmatori di scrivere codice parallelo e sfruttare appieno il potenziale di calcolo delle GPU.

%% Immagini sull'architettura GPU


%% Spiegazione CUDA

%% esempio programma cuda


%% thread blocchi e griglie cuda

%% OpenCL

%% Spiegare OpenCL/OpenGL -> Vulkan



% 2. **Stato dell'arte:**
%    - Fornisci una panoramica sullo stato dell'arte del GPU computing, inclusi i progressi recenti, le tecnologie chiave e le applicazioni più rilevanti.
%    - Descrivi le principali architetture GPU (ad esempio, NVIDIA CUDA, AMD Stream) e le loro caratteristiche.

% 3. **Architettura delle GPU:**
%    - Approfondisci l'architettura delle GPU, spiegando i componenti chiave come i core di calcolo, la gerarchia della memoria e i meccanismi di parallelismo.
%    - Illustra come l'architettura delle GPU si differenzia da quella delle CPU tradizionali.

% 4. **Programmazione parallela:**
%    - Descrivi i concetti fondamentali della programmazione parallela, compresi i thread, i blocchi, e le griglie in CUDA o strutture simili nelle altre architetture.
%    - Spiega come scrivere codice parallelo e ottimizzato per le GPU, evidenziando le sfide e le best practice.

% 5. **Applicazioni di GPU computing:**
%    - Dedica un capitolo alle diverse applicazioni del GPU computing in settori come il machine learning, la simulazione scientifica, la grafica, il calcolo finanziario, ecc.
%    - Esamina studi di caso o progetti che dimostrino l'efficacia delle GPU in questi contesti.

% 6. **Strumenti e sviluppo software:**
%    - Presenta gli strumenti e i framework utilizzati per lo sviluppo di applicazioni GPU, come CUDA Toolkit, cuDNN per il deep learning, ecc.
%    - Illustra le best practice per il debugging e la profilazione del codice su GPU.

% 7. **Risultati e sperimentazioni:**
%    - Riporta i risultati delle tue sperimentazioni o studi di caso, dimostrando l'impatto del GPU computing sulle prestazioni delle applicazioni.
%    - Utilizza grafici e dati per supportare le tue conclusioni.

% 8. **Discussione:**
%    - Discuti dei vantaggi e dei limiti del GPU computing.
%    - Rifletti sulle sfide e le possibili evoluzioni future di questa tecnologia.

% 9. **Conclusioni:**
%    - Riassumi i punti chiave e le scoperte principali della tua tesi.
%    - Sottolinea l'importanza delle tue ricerche e le possibili applicazioni o sviluppi futuri.

\section[Programmazione GPGPU]{Programmazione GPGPU}


% % use [][] to prepend/postpone text to the citation
% \cite[Hi][Goofy]{IEEEexample:article_typical}

% \si{\kilo\gram\per\second}

% % generic figure
% \begin{figure}[h]
% \centering
% % \includegraphics[width=.9\linewidth]{images/logo/logoPoliTo_with_name_wrong.png}
% \caption{Hi}
% \label{fig:hi}
% \end{figure}

% % use [] to set name for ToC
% \section[Extremely long name with manual linebreak which otherwise would not fit the page]{Extremely long name with manual linebreak\\which otherwise would not fit the page} % ok with fontsize=12pt

% % list
% \begin{enumerate}
%     \item A
%     \item B
%     \item C
% \end{enumerate}

% % minipage to put two images in the same figure
% \begin{figure}[h]
%     \centering
%     \begin{minipage}[t]{.49\linewidth}
%     \begin{figure}[H]
% 	\centering
% 	% \includegraphics[width=\linewidth]{images/logo/logoPoliTo_with_name_low_quality.jpg}
% 	\caption{HI}
% 	\label{fig:c}
%     \end{figure}
%     \end{minipage}
%     \hfill
%     \begin{minipage}[t]{.49\linewidth}
%     \begin{figure}[H]
% 	\centering
% 	% svg inclusion, requires inkscape
% 	% \includesvg[width=\linewidth]{images/artificial_neural_network.svg}
% 	\caption{SVG}
% 	\label{fig:svg}
%     \end{figure}
%     \end{minipage}
% \end{figure}

% \begin{table}[]
%     \centering
%     \setcellgapes{3pt}
%     \makegapedcells
%     \begin{tabular}{|c|c|c}
%     \hline
%     ReLU & $f(x) = \begin{cases}
% 	0 & \text{for } x \le 0\\
% 	x & \text{for } x > 0\end{cases}$ \\ \hline
%     Softmax & $f_i(\vec{x}) = \dfrac{e^{x_i}}{\sum_{j=1}^J e^{x_j}} i = 1, ..., J$ \\ \hline
%     tanh & $f(x)=\tanh(x)=\dfrac{(e^{x} - e^{-x})}{(e^{x} + e^{-x})}$ \\ \hline
%     \end{tabular}
%     \caption{Examples of activation functions, operating either element-wise or vector-wise, depending on the function}
%     \label{tab:activation_functions}
% \end{table}

% \begin{equation}
%     \label{eq:fully_connected}
%     output = f_{activation}\left(\displaystyle\sum_{\#neurons} input_i + bias\right)
% \end{equation}

% \begin{table}
%     \centering
%     \begin{adjustbox}{width={0.9\textwidth},totalheight={\textheight},keepaspectratio} % needed if the table overflows the margins, requires adjustbox package
%     \setcellgapes{3pt}
%     \makegapedcells
%     \begin{tabular}{|c|c|}
%     \hline
%     MSE / L2 Loss / Quadratic Loss & $\dfrac{\sum_{i=1}^{N} \left(y_i - \hat{y}_i\right)^2}{N}$ \\ \hline
%     \makecell{(Binary) Cross Entropy \\ (average reduction on higher dimensions)} & $\dfrac{\sum_{i=1}^{N} \sum_{j=1}^{C} \hat{y}_i \log\left(y_{i,j}\right)}{N}$ \\ \hline
%     \makecell{Categorical Cross Entropy \\ (sum reduction on higher dimensions)} & $- \sum_{i=1}^{N} \hat{y}_i +  \log\left(\sum_{i=1}^{N} \sum_{j=1}^{C} y_{i,j}\right)$ \\ \hline
%     \end{tabular}
%     \end{adjustbox} % must be closed before label and caption
%     \caption{$y$ is the output of the network, $N$ is the batch size multiplied by the number of outputs (e.g. pixels), $C$ is the number of classes and $\hat{y}$ is the correct output.}
%     \label{tab:loss_functions}
% \end{table}


% \begin{algorithm}
% \caption{Adam optimizer algorithm. All operations are element-wise, even powers. Good values for the constants are $\alpha=0.001, \beta_1 = 0.9, \beta_2 = 0.999, \epsilon = 10^{-8}$. $\epsilon$ is needed to guarantee numerical stability.}
% \label{alg:adam_optimizer}
% \begin{algorithmic}[1]
% \Procedure{Adam}{$\alpha, \beta_1, \beta_2, f, \theta_0$}
% \LineComment{$\alpha$ is the stepsize}
% \LineComment{$\beta_1, \beta_2 \in \left[0, 1\right)$ are the exponential decay rates for the moment estimates}
% \LineComment{$f\left(\theta\right)$ is the objective function to optimize}
% \LineComment{$\theta_0$ is the initial vector of parameters which will be optimized}
% \LineComment{Initialization}
% \State $m_0 \gets 0$
% \Comment{First moment estimate vector set to 0}
% \State $v_0 \gets 0$
% \Comment{Second moment estimate vector set to 0}
% \State $t \gets 0$
% \Comment{Timestep set to 0}
% \LineComment{Execution}
% \While{$\theta_t$ not converged}
% \State $t \gets t + 1$
% \Comment{Update timestep}
% \LineComment{Gradients are computed w.r.t the parameters to optimize}
% \LineComment{using the value of the objective function}
% \LineComment{at the previous timestep}
% \State $g_t \gets \nabla_\theta f\left(\theta_{t - 1}\right)$
% \LineComment{Update of first-moment and second-moment estimates using}
% \LineComment{previous value and new gradients, biased}
% \State $m_t \gets \beta_1 \cdot m_{t - 1} + \left( 1 - \beta_1 \right) \cdot g_t$
% \State $v_t \gets \beta_2 \cdot v_{t - 1} + \left(1 - \beta_2 \right) \cdot g_t^2$
% \LineComment{Bias-correction of estimates}
% \State $\hat{m}_t \gets \dfrac{m_t}{1 - \beta_1^t}$
% \State $\hat{v}_t \gets \dfrac{v_t}{1 - \beta_2^t}$
% \State $\theta_t \gets \theta_{t - 1} - \alpha \cdot \dfrac{\hat{m}_t}{\sqrt{\hat{v}_t} + \epsilon}$
% \Comment{Update parameters}
% \EndWhile
% \State \textbf{return} $\theta_t$
% \Comment{Optimized parameters are returned}
% \EndProcedure
% %\end{small}
% \end{algorithmic}
% \end{algorithm}

% % bullet points
% \begin{itemize}
%     \item A
%     \item B
%     \item C
% \end{itemize}

