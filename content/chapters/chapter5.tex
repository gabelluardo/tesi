\chapter{Analisi dei Benchmark}
\label{sec:analisys}

\section{Considerazioni sui Risultati}

% CUDA è ottimizzato per girare su NVIDIA e nonostante questo vulkan a volte va meglio

Considerando che i benchmark sono stati eseguiti su hardware prodotto da NVIDIA e con \gls{CUDA} che è un framework proprietario sviluppato da NVIDIA stessa, che Vulkan il abbia prestazioni paragonabili e in alcuni casi migliori di \gls{CUDA} è un dato sorprendente. Soprattutto per quanto riguarda i benchmark sulle moltiplicazioni di matrici in fig. \ref{fig:bench_f64} si può notare come l'esecuzione su dati a virgola mobile con doppia precisione sia ottimizzata meglio in \gls{CUDA}, anche se la differenza della media dei tempi è comunque nell'ordine del millisecondo. Se si prende in considerazione anche il trasferimento dei dati tra \textit{host} e \textit{device}, Vulkan risulta essere più performante su tutti i tipi di dato, dimezzando addirittura i tempi di esecuzione. Per quanto riguarda i dati sulla somma dei vettori, in tutti i benchmark Vulkan risulta meno efficiente rispetto a \gls{CUDA}, ma comunque con una differenza prestazionale non marcata. Questo è dovuto alla natura stessa del benchmark, poiché il caricamento del contesto e l'inizializzazione delle risorse thread richiede tempo per la quantità di elementi presi in esame, il tutto per eseguire solo un'operazione di somma: è indubbio che \gls{CUDA} sia ottimizzato in modo migliore da questo punto di vista. In contesti reali, si sarebbe eseguito un \textit{loop-unrolling} sul kernel per diminuire il numero dei thread rilasciati e quindi l'overhead dovuto alla gestione dei thread. Inoltre, i risultati ottenuti potrebbero essere influenzati dal fatto che i benchmark sono stati eseguiti su un singolo dispositivo e non su un cluster di \gls{GPU}, quindi non si può escludere che i risultati possano variare in base al numero di \gls{GPU} presenti nel sistema.

% vulkan buono per la portabilità e le prestazioni

I risultati ottenuti evidenziano che Vulkan può essere considerata un'alternativa valida a \gls{CUDA} per lo sviluppo di applicazioni di alto livello. Inoltre, Vulkan è un'\gls{API} \textit{open-source}, quindi non è legata a un singolo produttore e può essere utilizzata su hardware di diversi produttori. Questo è un vantaggio non trascurabile, poiché permette di sviluppare applicazioni che possono essere eseguite su hardware di vendor diversi senza dover riscrivere il codice. Un altro elemento importante da considerare è che mentre lo scope principale di \gls{CUDA} è il computing, quello di Vulkan è la grafica. Nonostante questo, i risultati ottenuti dimostrano che le ottimizzazioni di Vulkan lo rendono in grado di essere utilizzato anche per applicazioni di computing generale. Inoltre, Vulkan è in grado di estendere le funzionalità fornite tramite estensioni, per adattarsi a nuove esigenze e migliorare le prestazioni.

\section{Considerazioni sulla Development Experience}

% descrivere l'ambiente e le estensioni usate (nvidia insights, vscode, ssh)

Per quanto riguarda lo sviluppo dei benchmark, in pratica, può essere considerato come lo sviluppo di due applicazioni uguali in \gls{CUDA} e in Vulkan. Come ambiente di sviluppo è stato usato \textit{Visual Studio Code} collegato tramite \textit{ssh} alla macchina di test, in modo da poter scrivere il codice e compilarlo direttamente sul dispositivo. Per quanto riguarda le estensioni, è stato utilizzato NVIDIA Nsight \cite[]{NVIDIA:nsight} per il debug sia di \gls{CUDA} che di Vulkan e linter e formatter disponibili per i linguaggi. Si è usato il Vulkan SDK di LunarG \cite[]{KG:vulkan_sdk} per la compilazione e l'esecuzione, ma senza l'uso del \textit{Validation Layer}, per i controlli di validazione degli shader a compile time, data la semplicità degli shader sviluppati.

% i tool di sviluppo sono buoni

Per lo sviluppo del benchmark di Vulkan, l'ecosistema Rust fornisce, out of the box, una serie di strumenti che semplificano e uniformano lo sviluppo:

\begin{itemize}
    \item \textbf{Cargo}: package manager che permette di gestire le dipendenze e di compilare il codice
    \item \textbf{rustfmt}: formatter per il codice sorgente
    \item \textbf{clippy}: linter per il codice sorgente
    \item \textbf{rust-analyzer}: \gls{LSP} per l'autocompletamento e la navigazione del codice
    \item \textbf{crates.io} e \textbf{docs.rs}: repository di librerie e documentazione
\end{itemize}

L'ecosistema Rust e i tool offerti offrono benefici in termini di esperienza di sviluppo al programmatore, permettendo di concentrarsi sullo sviluppo del codice e piuttosto che sulla configurazione dell'ambiente di sviluppo. Quindi con la sola esclusione di NVIDIA Nsight, si è stato in grado di sviluppare facilmente applicazioni di alto livello, usando soltanto tool open-source.

\newpage

Per quanto riguarda lo sviluppo di \gls{CUDA}, non si può dire lo stesso. Partendo dal fatto che \gls{CUDA} di per sè è un ambiente proprietario, ad esempio, del compilatore \gls{NVCC} non si possono conoscere le ottimizzazioni che vengono applicate al codice. Inoltre, essendo praticamente un ecosistema che si rifà a C++, la mancanza di tool standardizzati e ufficialmente mantenuti dalla \textit{Standard C++ Foundation} ha portato a una frammentazione dell'ecosistema, che obbliga lo sviluppatore a scegliere a monte tra i vari sistemi di sviluppo come, ad esempio, \textit{make}, \textit{cmake} o \textit{ninja}, lo stile di formattazione da usare e uno tra i linter disponibili. Inoltre, la mancanza di un package manager ufficiale obbliga il programmatore a gestire le dipendenze manualmente, con la conseguente necessità di conoscere a priori le librerie da utilizzare e come installarle, col rischio di avere un sistema di building non riproducibile e poco portabile.

A parte queste considerazioni, una volta impostati i tool e l'ambiente, lo sviluppo con \gls{CUDA} è molto più facile e veloce rispetto a Vulkan. L'avere un
compilatore che estende il linguaggio C++ con delle keyword che isolano il
codice dei kernel da quello \textit{host}, permette di scrivere codice molto più simile a quello che si scriverebbe in C++ puro. Quanto visto nel capitolo \ref{sec:benchmark} per l'uso dei \textit{generics} è esplicativo in tal senso. Inoltre, il fatto che \gls{CUDA} sia un framework proprietario permette di avere un supporto molto più affidabile rispetto a Vulkan, che è un'\gls{API} open-source.

% vulkan è verboso

Per quanto riguarda Vulkan, la sua natura di \gls{API} a basso livello rende lo sviluppo più complesso rispetto a \gls{CUDA}. La necessità di gestire manualmente la creazione e allocazione delle risorse, come ad esempio le \textit{pipeline} o i \textit{buffer}, rende il codice molto più verboso e complesso. Sebbene parte di questa complessità sia attenuata da Rust, che si occupa di della distruzione delle risorse e la gestione della memoria, in confronto allo sviluppo con \gls{CUDA}, lo sviluppo con Vulkan risulta più complesso e richiede sicuramente una maggiore attenzione ai dettagli. Inoltre, la necessità di dover scrivere gli shader in un linguaggio specifico, come \gls{GLSL}, richiede di dover conoscere un linguaggio di programmazione in più rispetto a \gls{CUDA}, che permette di scrivere i kernel praticamente in C++.
Questi problemi potrebbero essere risolti scrivendo libreria che elevino il grado di astrazione per Vulkan e che permettano di scrivere codice più simile a quello che si scriverebbe in \gls{CUDA}, ma questo è sicuramente uno sforzo non da poco, che non è stato affrontato ai fini del benchmark.

% CUDA è più semplice ma scrivere in C++ è difficile

In conclusione, sia \gls{CUDA} che Vulkan hanno punti di forza e debolezze per quanto riguarda la \gls{GPGPU}: quelle di \gls{CUDA} derivano in larga parte dall'integrazione con C++ e il suo ecosistema, mentre, quelli di Vulkan dalla sua natura di \gls{API} a basso livello, open-source, più improntata alla grafica che al computing. Per quanto riguarda lo sviluppo di applicazioni multi-piattaforma, comunque, Vulkan è la scelta obbligata e i risultati ottenuti dimostrano che è in grado di competere con \gls{CUDA} in termini di prestazioni. Sebbene siano in sviluppo librerie per compilare direttamente codice Rust in \gls{SPIR-V}, come \textit{rspirv} \cite[]{KG:rspirv}, che potrebbe rendere lo sviluppo di applicazioni Vulkan più semplice, al momento non è possibile fare un confronto diretto tra le due \gls{API} in termini di sviluppo. Inoltre l'attenzione che sta mettendo NVIDIA su \gls{CUDA} e il suo ecosistema, come dimostrato dal recente rilascio di \textit{NVIDIA HPC SDK} \cite[]{NVIDIA:hpc_sdk}, fa intendere che \gls{CUDA} può essere ancora considerata la scelta principale per lo sviluppo di applicazioni di \gls{GPGPU}.

