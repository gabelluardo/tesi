\chapter{Conclusioni}
\label{sec:conclusion}

L'obiettivo di questo lavoro di tesi era la comparazione delle performance di due sistemi per il \gls{GPGPU} computing, \gls{CUDA} e Vulkan. Come dimostrato dai dati e dall'esperienza di sviluppo, per quanto \gls{CUDA} è ancora la prima scelta per sistemi di computazione eterogenea per \gls{GPGPU}, Vulkan è un'alternativa valida, in molti casi preferibile, e, per sistemi che non supportano \gls{CUDA}, l'unica scelta possibile. Dato che le performance di Vulkan sono molto vicine a quelle di \gls{CUDA}, la scelta tra i due dipende da altri fattori: la complessità del codice, la disponibilità di librerie e la facilità di sviluppo. Inoltre, Vulkan è una tecnologia più recente, rispetto a \gls{CUDA}, che continua ad evolversi e migliorarsi, quindi è possibile che in futuro diventi la tecnologia preferita per il \gls{GPGPU} computing.

La scelta di usare Rust in combinazione con Vulkan si è corretta. Rust è un linguaggio di programmazione moderno e sicuro, che permette di scrivere codice efficiente e mantenibile, senza rinunciare alle performance. Il vasto ecosistema di librerie e documentazione che si è sviluppato attorno a Rust è un ulteriore vantaggio per gli sviluppatori, soprattutto per quanto riguarda lo sviluppo web. Grazie alla \gls{FFI} di Rust anche l'integrazione con \gls{CUDA} è stata agevole e, al netto della preliminare fase di configurazione di building e linking delle librerie \gls{CUDA}, non ci sono stati impedimenti di sorta. Grazie alla politica \textit{zero cost abstraction} di Rust, l'integrazione con \gls{CUDA} è stata molto efficiente senza al presenza di alcun overhead nell'applicativo.

In futuro, sarebbe interessante esplorare le performance di entrambi i framework su sistemi multi-GPU, per vedere se Vulkan è in grado di sfruttarne in modo migliore l'hardware rispetto a \gls{CUDA}. L'idea di poter anche usare Vulkan in combinazione con \gls{CUDA} per sfruttare le potenzialità di entrambe le tecnologie potrebbe portare a risultati molto interessanti.
L'ostacolo principale che, ad oggi, impedisce il diffondersi di Vulkan per la \gls{GPGPU} è verbosità nello scrivere codice: l'inizializzare delle risorse Vulkan, impostare il trasferimento di memoria con i buffer e sincronizzare i comandi, di certo inficia la scrittura di codice facilmente comprensibile e mantenibile. Sarebbe interessante sviluppare librerie e framework per semplificare lo sviluppo di applicazioni \gls{GPGPU} con Vulkan, in modo da rendere questa tecnologia più accessibile agli sviluppatori. L'idea di poter compilare Rust direttamente in \gls{SPIR-V}, senza passare per il \gls{GLSL}, ed eseguire il codice direttamente su Vulkan, potrebbe essere la killer feature che renderebbe Vulkan una valida alternativa a \gls{CUDA} per il \gls{GPGPU} computing.
