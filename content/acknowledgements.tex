% acknowledgements


Ogni viaggio inizia con un primo passo, e oggi posso dire di aver compiuto un passo importante. Guardo indietro e vedo il lungo sentiero che ho percorso per arrivare fin qui, un cammino troppo insidioso da intraprendere senza il sostegno delle persone che mi sono state vicine, condividendo con me difficoltà e gioie. Vorrei esprimere la mia profonda gratitudine a tutte queste persone, i cui contributi hanno reso possibile il raggiungimento di questo grande traguardo.

Un grazie di cuore va al Prof.~Giovanni Malnati, relatore di questa tesi, per avermi guidato e supportato in questo percorso, per avermi dato la possibilità di esplorare nuovi orizzonti e per avermi trasmesso la passione per la ricerca, l'innovazione e soprattutto per Rust. Grazie per avermi dato la possibilità di lavorare a questo progetto.

Un doveroso ringraziamento va ai miei azionisti di maggioranza, senza i quali tutto questo non sarebbe stato possibile: Madre e Padre. Supporto finanziario ed emotivo che mi ha permesso di iniziare e concludere (quasi incolume) questo viaggio.

A Sorella, che, dalle tagliatelle con la salsa cruda ai pastel de Nata, è stata il supporto calorico e caloroso di cui avevo bisogno, ``sem queijo, sem leite''.

Alla famiglia di giù, che ha deciso di spargersi per l'Italia: ``scendere per le feste'' è di per sè una festa solo perché ci siete tutti voi.

Alla famiglia che ho trovato su e mi ha fatto sentire a casa, sopravvivendo a tutti i miei drammi e dad joke: grazie per essere stati compagni insostituibili di questo viaggio. Che fosse una pampanella, un frico, una carbonara, una fregola, una pasta zucchine e Philadelphia, un sushi, una genziana o semplicemente un caffè, quei momenti passati insieme hanno reso meno gravosa questa scalata, che in solitaria sarebbe stata insostenibile.

Agli altri due moschettieri che da 20 anni mi supportano e sopportano: ce l'abbiamo fatta, adesso siamo un trio di dottori.

A voi tutti un grazie dal più profondo del mio cuore, mi avete fatto capire perché ``amicizia'' ha la stessa radice di ``amore''.

Al team Policumbent e soprattutto ai nevadini: abbiamo condiviso sia esperienze fantastiche che momenti terrificanti. Ogni istante lo custodirò gelosamente e rimarrò per sempre legato a all'avventura che abbiamo intrapreso insieme. Bikes on the road!

Voglio dedicare questo lavoro di tesi, e tutto il percorso che oggi si conclude, alla persona che per prima in assoluto ha creduto in me (anche se mi avrebbe voluto avvocato). Nonna Pina, so che non potrai leggere o sentire queste parole, ma sarai l'unica ad avere il nome tra queste pagine. In qualsiasi tempo e luogo tu sia, spero possa arrivarti chiaro che ``nasca bedda'' si è appena laureato ingegnere.

Concludo lasciando un messaggio al me futuro, che sicuramente col tempo rileggerà queste parole: ``lo so, la salita è stata dura e da qui la vista è meravigliosa, ma non non è finita. Questo non è che il primo campo base, la cima è lassù che ci aspetta e io ho già la bandiera in mano''.
